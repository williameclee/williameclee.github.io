\documentclass[a4paper, 12pt, mla]{homework}

\title{Carved in Stone}
\subtitle{ENGL 101-011\ \ First-Year Composition \linebreak Project 1: A Literacy Narrative}
\author{En-Chi (William) Lee}
\authorline{Originally submitted on September 25, 2023}
\mail{williameclee@arizona.edu}

\usepackage{Alegreya, AlegreyaSans}


\begin{document}
\maketitle


`Look carefully into the rocks around you!' 
Professor Chen exclaimed to the gathered students, his words resonating through the rugged gorge.  
`Every natural process leaves its mark in the rocks, like footprints of ages past. 
	Pay attention, and this valley's story will unveil itself before your eyes.' 

Rocks, these ancient scribes, have been documenting the history of nature for aeons, long before the first of humankind ever wandered the planet. 
They shall persist in their duty long after the final tale is told by the bedside lamp's glow. 
Rocks endure. 
Will they tell of our history? 
What stories are we carving into them?

It was November 2021, when the Physical Geology class at National Taiwan University was drawing to a close. 
Professor Chen, the most respected scholar of Taiwanese geology, had led our class, along with my field partner Chou and myself, on a field expedition. 
After countless lectures on rock identification and deciphering the messages hidden within Earth's geological structures, our excitement was audible as we boarded the bus. 
We journeyed northward along the coastal freeway; 
the Seaboard Mountain Range loomed to our left, while the majestic Pacific Ocean stretched infinitely to our right, tantalisingly close.

The bus halted near the No.\,10 Bridge, where a creek met the sea. 
It had no name, for not even the indigenous tribe residing in the mountains had ventured close enough to bother naming it--it was just a waterway among countless others that poured into the ocean at this lonely stretch of the freeway, and it was carelessly called the No.\,10 Bridge Creek. 
As we hiked upstream along the tranquil stream, the sound of chatters and passing vehicles on the road gradually subsided, yielding the stage to the gentle murmur of running waters. 
It developed into a repertoire accompanied by the rhythmic percussion of my laboured breaths as I scaled boulders, occasionally punctuated with trills of our shoes stirring the pebbles that lined the riverbank.

At last, we stopped in a valley carved deep by the creek, kilometres removed from the deafening civilisation. 
Despite being a stop during Professor Chen's annual expeditions, I was doubtful that this sanctuary had echoed human speech in the past decade.

`Look carefully into the rocks around you!' 
Professor Chen's call resounded once more in the valley. 
Swift surveys by us soon identified the predominant rock type comprising the valley walls as conglomerates--sedimentary rocks resembling inorganic chocolate chip biscuits. 
In a rushing river, water transported stone fragments, rounding their edges as it flowed. 
As the stream meandered towards peace, these pebbles descended to the riverbed, buried within layers of sand and silt. 
Over time, the accumulating strata of these fragments consolidated under their weight, transforming into conglomerates. 
In fact, the very pebbles and boulders beneath our feet, shed from the pale cliffs or carried by the water from a nearby extinct volcano, would one day contribute to yet another layer of conglomerate. Each rock bore a unique tale, with their composition and form unveiling the story of this valley. 

A peculiar stone fragment on the riverbed captivated me. 
It stood apart from the surrounding rocks that had thus far met our gaze. 
It bore the hallmarks of a conglomerate, but possessed a distinctiveness unprecedented elsewhere in the valley. 
Comprising ivory-white pointed fragments wrapped in a grey matrix, it sparked a puzzling sense of d\'ej\`a vu. The clasts displayed a peculiar uniformity in size and shape. 
Surely, I had encountered a specimen of this kind before, but the specifics fled me.

`This must be volcanic breccia,' 
Chou declared, accepting the shard I offered. 
`See the angularity of the gravel within and the presence of tiny crystals. 
	It's a tell-tale sign of volcanic origin.' 
Indeed, volcanic breccia served as an igneous relative of conglomerate, sharing visual similarities but distinguished by its jagged constituents. 
Other igneous boulders scattered across the valley lent credence to his argument. 

`It cannot be,' I countered. 
`Volcanoes in this region produce rocks of dark hue, entirely distinct from this specimen. 
	I say this is no more than a common sedimentary rock, with clasts that have been insufficiently abraded.'

Unable to sway each other, we parted ways in search of evidence to support our respective claims. 
Upon closer examination of the valley walls, I chanced upon a feature known as the Bouma sequence. This revelation indicated that the conglomerate had not been deposited in a river as we had initially thought, but during a submarine landslide--a sedimentary condition in which breccia could form. 
My theory was proven. 
I held the stone fragment high triumphantly and called out Chou's name.

Suddenly, Professor Chen interjected. 
`Stop carrying that piece of floor tile as if it is precious, you morons.' 
His tone was patient and indulgent, as if addressing five-year-olds.

Ah, it became clear in an instant. 
This `rock' was exactly identical to the floor tiles adorning my grandmother's home. 
No wonder it seemed familiar. 
But how had this fragment journeyed to this place? A multitude of questions surged through my mind. 
Could there be a settlement upstream, or had it been carried here by the violent winds of a typhoon? 
Might more such fragments--perhaps roofing tiles or bricks--lie concealed in this valley? 
The discovery of such an artefact in this remote wilderness defied all my expectations. 
And yet, it seemed to belong here, appearing entirely... natural.

No. 
It \textit{was} natural. 
Ultimately, \textit{Homo sapiens} were merely another force of nature. 
In a few tens of thousands of years, this shard would be compressed and cemented into a part of the rock stratum--a trace fossil, a testament to our passage in the deep geological history.

Despite crafting a stratum marked by odd metal-rich anomalies and serving as relics of a time marred by the dwindling diversity of fossils, 
`overall, most of the artefacts of humanity would be lost in less than a thousand years,' as I later read in The Human Planet by Simon L. Lewis and Mark A. Maslin. 
Nevertheless, this humble shard of floor tile might endure as one of the rare remnants of human existence, distinguishable without the need for advanced tools. 
Could it be that, in a distant future beyond reckoning, geologists of an entirely different species might stumble upon this enduring piece of tiling, forever cemented in rock? 
Might they, too, be captivated by its distinctiveness amid the surroundings, tracing its origin to a foreign species of bygone days?

Perhaps they shall. 
They will dissect the composition of this `rock' and deduce that the uniform clasts, devoid of any naturally occurring minerals, must result from intelligent design. 
How, then, will they interpret our story? 
Will they craft a narrative of a mysterious and technologically advanced society that sought to reshape the Earth to suit its desires, or will they dismiss us as merely one of the many civilisations that ever existed in the stream of time, the same way the Pacific Ocean engulfed a nameless creek? 
Or, as they gather more pieces of this puzzle from rocks of the same age, will they weave a cautionary fable of a species recklessly striving to play the role of gods, only to bring about a biological catastrophe and accelerate its own demise? 

Our future will be immortalised in stones. What tales do we want the rocks to impart to those who come after us?

\end{document}