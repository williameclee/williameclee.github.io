\documentclass[a4paper, 12pt, mla]{homework}

\title{Annotated Bibliography of the Genre of Absurd Gameplay Advertisements}
\subtitle{ENGL 101-011\ \ First-Year Composition \linebreak Project 2: Annotated Bibliography of a Genre}
\author{En-Chi (William) Lee}
\authorline{Originally submitted on October 23, 2023}
\mail{williameclee@arizona.edu}

\usepackage[scale=0.85]{FiraMono}
\usepackage{Alegreya, AlegreyaSans}


\begin{document}
\maketitle

\begin{abibliography}
\abitem{Ihsan, R. K., \& Junaidi, F. (2022). A Content Analysis of Advertising Appeal in Free to Play Mobile Game Ads. In \textit{Symposium of literature, culture, and communication (SYLECTION) 2022}. doi: \doi{10.12928/sylection.v1i1.11277}}
	\begin{itemize}
		\item \textit{Summary}: 
			This journal article examines the advertising appeals utilised in Free-to-Play (F2P) mobile game advertisements, analysing 100 advertisements on YouTube spanning the popular F2P mobile game genres from 2017-2021. 
			The study reveals that rational appeals featured in merely 20\% of the adverts, while emotional appeals dominated with an 80\% presence. 
			Four of the emotional appeals, categorised as `positive' (humour and adventure) and `negative' (guilt \& sadness and violence \& fear), were each reflected in approximately $15\%-18\%$ of the analysed content.
			The researchers argue that F2P mobile games, categorised as `low-involvement' products due to the minimal risk associated with downloading them, benefit more from emotionally provocative appeals to stand out among a sea of similar competitors.
		\item \textit{Quotables}:
			\begin{itemize}
				\item `...emotional appeal is often used in low-involvement products because it can lift commodities from the level of similarity and position them differently in customers' minds.'
				\item `...the attractiveness of advertisements containing provocative messages in F2P mobile game advertisements was 62.40\%, dominated using humour (18.67\%), guilt and sadness (18.17\%), and violence and fear (15.66\%).'
			\end{itemize}
		\item \textit{Volcabularies}: free-to-play (F2P) mobile game, low-involvement, emotional appeal, fear of missing out (FOMO)
		\item \textit{Remarks}:
			\begin{itemize}
				\item While the study sheds light on the predominant use of emotional appeals in YouTube advertisements for F2P mobile games, it leaves a gap regarding how the usage of these appeals may vary across different platforms. 
					For instance, an in-game advertisement using rational appeal about the absence of in-game ads in the advertised game, which I have personally encountered several times, may resonate stronger with the viewer.
				\item % The provision of statistical data is commendable as it challenges the common tendency to remember only the more peculiar ads, usually falling into humour, sexual, or guilt \& sadness appeals. 
					The reported 15\% prevalence of violence \& fear in the ads does not align with my personal experiences, underscoring algorithm-driven variability in commercial exposure across different individuals. 
%					This might contribute to the scarcity of research on this phenomenon.
			\end{itemize}
	\end{itemize}


\abitem{Knezovic, A. (2023, April). Homescapes Ads Analysis: Truth about Misleading Ads. \textit{Udonis}. Retrieved October 20, 2023, from \url{https://www.blog.udonis.co/mobile-marketing/mobile-games/homescapes-analysis/}}
	\begin{itemize}
		\item \textit{Summary}:
			This article delves into the deceptive advertising strategies employed by the mobile game \textit{Homescapes}, a casual `match-3' game from Playrix's `Scapes' series. 
			Knezovic categorises \textit{Homescapes} digital commercials based on their themes: mini-game ads, narrative/storytelling, decoration, and match-3. 
			A notable pattern in these advertisements is the habitual failure of the `player,' seemingly aimed to engage viewers. 
			The author concludes that these promotional materials appeal to a broader audience appreciative of the game's `meta elements' such as decorating and storytelling, rather than focusing on its gameplay mechanics. 
			Players interested in these meta elements have a higher `affinity' towards the game. 
			This advertising strategy broadened the game's user base, boosting its downloads and visibility in app stores, thus leading to more organic installs.
		\item \textit{Quotables}:
			\begin{itemize}
				\item `Playrix’s games don’t primarily target the ``vanilla'' Candy Crush type audience, but a broader one, that’s further away in terms of affinity. More specifically, audiences who are interested in story adventure games, hidden object games, and even resource management games.'
				\item `Reaching a broader audience with misleading ads comes at a cost. Many players that were targeted with these ads might end up enjoying Homescapes -- it’s a fun game. However, some players might be disappointed. That results in bad reviews -- Homescapes gets plenty of them.'
			\end{itemize}
		\item \textit{Volcabularies}: meta element, audience affinity, organic user acquisition (ASO)
		\item \textit{Remarks}:
			This piece sheds light on the rationale behind structuring misleading commericlas in a particular manner, aiming to showcase the games' meta elements, which tend to resonate with a wider audience, instead of advertising the generic mechanics.
			This deviates from the self-assessments of gamers surveyed in Nexter (2023).
	\end{itemize}


\abitem{Mago, Z. (2020). Fake-vertising and Mobile Games: Case Study of `Pull the Pin' Ads. \textit{Communication Today}, \textit{11}(2), 132-147. Retrieved from \url{https://communicationtoday.sk/fake-vertising-and-mobile-games-case-study-of-pull-the-pin-ads/}} 
	\begin{itemize}
		\item \textit{Summary}:
			This article delves into the modern trend of employing `fake game ads' that showcase fictional or misleading gameplay footage to entice users. 
			The emergence of this deceptive strategy traces back to mid-2019, with \textit{Hero Wars} by Nexters being one of the initiators and \textit{Homescapes} becoming a notable adopter.
			
			Specifically, the author highlights that the `pull the pin' mechanics, where a player is tasked with pulling a pin to accomplish a goal, is featured in the promotional materials of 106 games on Google Play. 
			However, a study of 10 games discloses that 80\% of them do not embed this mechanic in their central gameplay. 
			While this strategy tends to taint ratings over time, a certain segment of users has developed a liking for the misleadingly advertised games. 
			Intriguingly, the `pull the pin' concept, originally created only for the commercials, spurred the creation of new games centred on this mechanic, thereby establishing it as a subgenre.
		\item \textit{Quotables}:
			\begin{itemize}
				\item `...gamers described the difference between the in-game and playthrough footage used in advertising at game events like E3 as absurd, and themselves subsequently filed a lawsuit against the companies Sega and Gearbox.'
				\item `The impact of `pull the pin' fake ads on `fake-vertised' games' ratings was mostly negative, indicating that over one-fifth of all-time negative ratings could possibly have been caused directly by using `fake gameplay ads', and negatively affect conversion rates in the long term.'
			\end{itemize}
		\item \textit{Volcabularies}: fake-vertising, pull the pin.
		\item \textit{Remarks}:
			\begin{itemize}
				\item The article thoroughly examines the evolution of `fake game ads,' although the term `non-core gameplay ads' proposed by Nexters (2023) might be more fitting in this scenario.
				\item The author's observation that these adverts tend to degrade the games' ratings doesn't elucidate why the tactic remains prevalent.
				\item The article also illuminates that mobile game players exhibit a higher tolerance towards these advertisements than the widespread objections seen when AAA game trailers misrepresent gameplay.
			\end{itemize}
	\end{itemize}


\abitem{Nexters. (2021, December). Hero Wars Become a legend Ad 44 (Advertisement). \textit{YouTube}. Retrieved October 17, 2023, from \url{https://youtu.be/PST8UpktCIs?si=Oj9SAx-2uOKrMZ2I}}
	\begin{itemize}
		\item \textit{Summary}:
			In this ad, a half-naked anime character is trying to defeat a high-level skeleton monster boss. 
			He must increase his level, indicated by numbers above him, by taking down enemies with lower levels, and their level is added onto his upon defeat. 
			The character meets his demise by trying to attack an enemy whose level is higher than his and is brutally stabbed.
		\item \textit{Quotables and volcabularies}: (none)
		\item \textit{Remarks}:
			This is a classic example of the genre `absurd gameplay advertisements,' wherein the protagonist fails to overcome a relatively straightforward challenge or puzzle.
	\end{itemize}


\abitem{Nexters. (2023, April). \textit{Non-Core Gameplay Ads Research: What Game Ad Do You Want to be Turned into a Full Game?} (Tech. Rep.). Retrieved from \url{https://drive.google.com/file/d/1fQFLg-Qg0I-OWOJUq7UmBlyCwO3kT6Jx/ view?usp=sharing}} 
	\begin{itemize}
		\item \textit{Summary}:
			This research report reveals that 71\% of the surveyed Americans have encountered game advertisements containing misleading content, and nearly 75\% of them find the actual gameplay less engaging in comparison. 
			Despite this, one-third to a half of the users who download games based on misleading promotional materials continue to play them.
			
			The respondents believe such advertisements aim to boost download numbers for otherwise uninspiring games, enhancing in-app purchase revenues for game companies. 
			The mechanics and gameplay advertised are cited as the primary appeals for downloading, with puzzles, \textit{Hero Wars}, and `Scapes' games like \textit{Gardenscapes}, as mentioned by Mago (2020), identified as the most underwhelming games when compared to their promotional materials.
		\item \textit{Quotables}:
			\begin{itemize}
				\item `...With companies adding mini-games with the mechanics from the ads, such advertising is no longer misleading... [this experience's] just not the core one.'
				\item `Most gamers believe developers resort to such ads to increase the number of downloads and revenue from in-app purchases and to hype up and underwhelming game, making it seem new and appealing.'
				\item `Gamers across the regions are most attracted by gameplay and mechanics in misleading ads and least -- by the new features and user interface.'
			\end{itemize}
		\item \textit{Volcabularies}:
			non-core gameplay ads, `Scapes' games
		\item \textit{Remarks}: 
			\begin{itemize}
				\item The research is led by Nexters, pinpointed by Mago (2020) as likely the pioneer in deploying misleading advertising tactics.
			\item Conveniently, the authors favour these `non-core gameplay ads,' claiming them as `no longer misleading.' 
				Despite this, the source is valuable as it provides statistics and sheds light on the perceptions of the advertisers behind these campaigns, irrespective of the authenticity of the data.
			\end{itemize}
		\end{itemize}
		
		
\abitem{Voodoo. (2018, October). If you reach level 2 you're legally a pumpkin (Advertisement). \textit{Instagram}. Retrieved October 17, 2023, from \url{https://www.reddit.com/media?url=https\%3A\%2F\%2Fi.redd.it\%2Fuwii60yz4zu11.jpg}}
	\begin{itemize}
		\item \textit{Summary}:
			This video advertisement adopts a meme-inspired format: the screen is split in two, with the left segment featuring a static image of a jack-o-lantern with an erratic expression, while the right segment displays the actual gameplay footage of the advertised game \textit{Helix Jump}. 
			The footage ends moments before the completion of level 2, reinforcing the caption: `If you reach level 2 you're legally a pumpkin.'
		\item \textit{Quotables}:
			\begin{itemize}
				\item `If you reach level 2 you're legally a pumpkin' 
				\item `The best arcade game!'
			\end{itemize}
		\item \textit{Volcabularies}: arcade game
		\item \textit{Remarks}:
			\begin{itemize}
				\item Contrary to the trend of misleading advertisements, this video accurately represents the gameplay mechanics of \textit{Helix Jump}. 
					This reflects how even legitimate games may harness the appeals typically associated with misleading advertisements.
				\item This advertisement subtly blends in on Instagram as a conventional meme post, rendering it less intrusive yet somewhat memorable due to its nonsensical caption.
			\end{itemize}
	\end{itemize}

\abitem{Wonder Group. (2022, November). Save the Doge (Advertisement). \textit{YouTube}. Retrieved October 17, 2023, from \url{https://youtu.be/CMg0Gc3mLRk?si=KcvMJjWL5UNrer5t}}
	\begin{itemize}
		\item \textit{Summary}:
			The advertisement depicts swarms of bees threatening two `doge' dogs, while the `player' attempts to block out the insects by drawing barriers. 
			The player fails to do so during the first three levels but prevails in the subsequent four.
		\item \textit{Quotables and volcabularies}: (none)
		\item \textit{Remarks}:
			\begin{itemize}
				\item This advertisement fits the `save the animals' theme recurring in numerous game advertisements. 
					They are often criticised for their irritating background music and the inappropriately shaped barriers drawn by the player.
				\item The narrative arc in this particular commercial, where the player initially fails but later succeeds, deviates from the customary trajectory in similar advertisements, often ending with the player's eventual fail.
			\end{itemize}
	\end{itemize}

\end{abibliography}

\end{document}